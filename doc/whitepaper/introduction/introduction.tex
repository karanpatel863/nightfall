\newpage

\part{Introduction}
\label{part:introduction}
\parttoc

\ \\
\\
\noindent
In this document we discuss \href{https://github.com/EYBlockchain/nightfall}{Nightfall}~\cite{nightfall} - an open source suite of tools designed to enable private token transactions over the public Ethereum blockchain.\\
\\
Nightfall is currently compatible with smart contracts which adhere to either of Ethereum's \href{https://github.com/ethereum/EIPs/blob/master/EIPS/eip-20.md}{ERC-20} or \href{https://github.com/ethereum/EIPs/blob/master/EIPS/eip-721.md}{ERC-721} token standards. These token standards have been chosen because of their prevalence in the Ethereum community today. Nightfall could indeed be expanded in the future to support other token standards.\\
\\
ERC-20 tokens are fungible in the sense that they can be subdivided and individual units are interchangeable.\\
ERC-721 tokens are non-fungible in the sense that they represent something unique.\\
\\
Nightfall was created by EY and released into the public domain in May 2019.





\section{zk-SNARKs}
\label{sec:zkSnarks}

\subsection{Overview}
A zero-knowledge succinct non-interactive argument of knowledge (zk-SNARK) enables a `Prover' to demonstrate that they have correctly performed a calculation on a set of inputs, without revealing some of those inputs.  This is useful in the context of privately transacting on a public blockchain. In a `traditional' (non-private, ERC-20) blockchain transfer, computations to update a sender and receiver's balances are performed publicly `on-chain' within the \texttt{transfer} function of a smart contract. This public \texttt{transfer} function requires the \texttt{to}, and \texttt{amount} values to be publicly input into the calculation.

Figure~\ref{fig:codeTransfer} depicts a public token transfer over Ethereum.

  \begin{figure}[h]
    \begin{lstlisting}

    /**
    * @dev Transfer token for a specified address
    * @param _to The address to transfer to.
    * @param _value The amount to be transferred.
    */
    function _transfer(address _to, uint256 _value) internal returns (bool) {
      require(_value <= balances[msg.sender]);
      require(_to != address(0));

      balances[msg.sender] = balances[msg.sender].sub(_value);
      balances[_to] = balances[_to].add(_value);
      emit Transfer(msg.sender, _to, _value);
      return true;
    }
     
    \end{lstlisting}
    \caption{An implementation of the ERC-20 `transfer' function}
    \label{fig:codeTransfer}
  \end{figure}

For private transfers, we require that the \texttt{to} and \texttt{amount} values to be kept private from the blockchain. This is non-trivial, because for any `traditional' computation within a smart contract, any values which are used within a calculation must necessarily be made public in order for all nodes to agree on the new states of the smart contract.
It therefore follows that we need to design alternative computations which are performed on-chain, in order to hide the inputs to a \texttt{transfer}. We will also need to re-think how `ownership' is ascribed to tokens. 
In a traditional ERC-20 contract the \texttt{balances} of each Ethereum address are public mappings, which is incompatible with our notion of privacy.
We require that the public key receiving tokens is not revealed by the transfer.

In order to keep the \texttt{to} and \texttt{amount} inputs private between the sender and receiver we use zk-SNARKs.
 The sender (or `Prover') runs a computation \textit{privately} on their own computer. They pass \textit{private} inputs into this computation and get a set of \textit{public} outputs which they  share with the blockchain. The public outputs appear as unreadable encrypted values to all observers; only the sender and receiver can interpret their full meaning. In order for these encrypted values to have `meaning' to all observers, the Prover also shares with the blockchain a corresponding `proof' of having correctly computed these outputs. Together this proof and these public outputs can be verified in such a way that everyone can be convinced that a pre-agreed calculation has been performed on a particular set of private inputs to produce the public outputs. In this case, the pre-agreed calculation represents a `transfer', and verification of the proof and public outputs can be unambiguously interpreted by observers as \textit{``somebody has submitted a binding intention to transfer funds to someone else''}.
For full details on the Prover's computation and the public outputs submitted to the blockchain, see Part~\ref{part:theProtocols}.

\subsection{Background}
Nightfall leverages the impressive mathematics of zk-SNARKs to achieve privacy.
In this paper zk-SNARKs are used in a blackbox manner and we do not discuss how they work.
We encourage interested readers to refer to explanations by
\href{https://z.cash/technology/zksnarks/}{Zcash}~\cite{whatAreSNARKs},
\href{https://medium.com/@VitalikButerin/zk-snarks-under-the-hood-b33151a013f6}{Buterin}~\cite{underTheHood},
and \href{https://blog.ethereum.org/2016/12/05/zksnarks-in-a-nutshell/}{Reitwiessner}\cite{zkSNARKsNutshell}.

Nightfall is implemented under a zk-SNARK by Groth and Maller~\cite{DBLP:conf/crypto/GrothM17}.
This zk-SNARK is simulation extractable, meaning that it is hard for an attacker to gain an advantage from other zk-SNARKs on the blockchain.
It has highly competitive efficiency in terms of proof size and verifier time.
On the other hand, it's security relies on the `Knowledge-of-exponent' assumption, which is a non-standard cryptographic assumption (in other words, although it is currently considered cryptographically secure, it is more likely to be broken than RSA or discrete-log).

\subsection{Efficiency}
Nightfall has a proving system for mint, transfer, and burn algorithms for both fungible and non-fungible assets which are proven secure using a zk-SNARK.
These protocols are described in Part~\ref{part:theProtocols}.
In Table~\ref{table:efficiency} we give an overview of the costs of each of these algorithms.

\begin{table}[H]
	\begin{center}
		\begin{tabular}{cccccccc}
			\toprule
      Algorithm &
      Constraints &
      \multicolumn{2}{c}{Key Sizes} & 
      \multicolumn{2}{c}{zk-SNARK Size} & 
      \multicolumn{2}{c}{Costs$^{4}$} \\
      
      \cmidrule(lr){3-4}\cmidrule(lr){5-6}\cmidrule(lr){7-8} & 
      
      &
      \makecell{Proving Key$^1$\\
      (Mbytes)} & 
      \makecell{Verification Key$^2$\\
      (bytes)} & 
      \makecell{Input$^3$\\
      (bytes)} & 
      \makecell{Proof$^3$\\
      (bytes)} &
      \makecell{Gas\\
      (M)} &
      US \$ \\ 

      \midrule
      
      \texttt{ft-mint} &
      $83,000$ &
      $32$ &
      $414$ & 
      $128$ &
      $256$ &
      $1.8$ &
      $10.21$ \\

      \texttt{ft-transfer} &
      $2,292,000$ & 
      $859$ & 
      $637$ &
      $352$ & 
      $256$ & 
      $2.7$ & 
      $14.77$ \\
      
      \texttt{ft-burn} &
      $1,075,000$ & 
      $401$ & 
      $478$ &
      $192$ &
      $256$ & 
      $1.7$ & 
      $9.49$ \\

      \texttt{nft-mint} &
      $83,000$ &
      $32$ &
      $510$ &
      $160$ &
      $256$ &
      $1.9$ & 
      $10.60$ \\

      \texttt{nft-transfer} & 
      $1,158,000$ & 
      $433$ & 
      $446$ &
      $224$ &
      $256$ &
      $2.1$ & 
      $11.52$ \\

      \texttt{nft-burn} &
      $1,075,000$ &
      $401$ & 
      $510$ &
      $224$ & 
      $256$ & 
      $1.8$ & 
      $10.03$ \\
      
      \bottomrule
		\end{tabular}
	\end{center}
	\caption{Efficiency overview of Nightfall.}
	\label{table:efficiency}
\end{table} 

$^1$ As calculated within ZoKrates during the `setup' stage. Files stored on a User's hard drive will be larger.\\
\\
$^2$ As calculated within ZoKrates during the `setup' stage. Each elliptic curve point of a Verification Key is encoded as either two \texttt{uint256} values (G1) or four \texttt{uint256} values (G2) within the Nightfall smart contracts, and so might be larger than shown here.\\
\\
$^3$ Based on their sizes when passed as calldata to the Nightfall smart contracts. Each input is a \texttt{uint256}. Each elliptic curve point of a Proof is encoded as either two \texttt{uint256} values (G1) or four \texttt{uint256} values (G2).\\
\\
$^{4}$ The price of Ether as at 30 May 2019 is $\$278$ (according to \href{https://coinmarketcap.com/}{Coin Market Cap}). Costs are based on a gas price of 20Gwei.\\
Gas measurements have been taken over an entire transaction. I.e. the gas cost shown for an nft-transfer (for example) reflects the overall cost of computation and storage across \textit{all} contracts involved in that transaction.\\
Verification of the zk-SNARK (within the Verifier contract) forms only a part of this total gas cost. The gas costs for the verification of a zk-SNARK \textit{alone} (within the Verifier contract) have not been measured, due to the difficulty in attributing the final `storage refund' at the end of a transaction between the Shield contract and the Verifier contract.


\subsection{Proof generation}

Nightfall uses \hyperref[sec:zokrates]{ZoKrates} to generate proofs. With that, the time it takes to generate a `proof' depends mainly on:
\begin{enumerate}
  \item the complexity of the computation being `proven'
  \item the computing resources being allocated to generate the proof
  \item advances in ZoKrates' efficiency for generating proofs
\end{enumerate}

\textbf{1.}
With zk-SNARKs, a `computation' is deconstructed into a series of arithmetic constraints. In Nightfall, the most constraints come from doing sha256 hashing from the leaf of a Merkle Tree to its root (see Part~\ref{part:theProtocols}, the Protocols, for details). This repeated sha256 hashing is performed during a `transfer' and a `burn' of both fungible and non-fungible token commitments. Most notably, for a fungible transfer, we need to perform this repeated hashing from leaf-to-root twice; for two leaves.\\
\\
At the moment, we have configured Nightfall to have a Merkle Tree of `depth' 33. That is, we have chosen to have a huge Merkle Tree with $2^{32} = 4,294,967,296$ leaves. That's capacity for over 4 billion token commitments before the Shield Contract runs out of space. If your use-case doesn't need to accommodate so many commitments for the rest of time, then you could reduce the depth of the tree in your implementation.\\
\\
A different hashing function would also reduce the time to generate a proof. Pedersen commitments were considered, because they have far fewer constraints.  However the repeated hashing from the leaf of a Merkle Tree to its root must also be performed \textit{on chain} (within the Shield Contract) with Nightfall. Given that there are no precompile contracts which reduce the gas costs for Pedersen commitment calculation, it is \textit{significantly} cheaper to use sha256.\\
\\
\textbf{2.}
The proof for a fungible transfer takes the most time to generate. Generally this can take anywhere from `a few minutes' to `around 10 minutes', depending on the power of the computer.\\
\\
\textbf{3.}
ZoKrates is continually updated with efficiency improvements. A newer image of ZoKrates could perhaps improve upon Nightfall's current proving times.
